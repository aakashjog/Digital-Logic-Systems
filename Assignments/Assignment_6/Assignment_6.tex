\documentclass[fleqn, a4paper, 11pt, oneside]{amsart}
\usepackage{exsheets, tasks}
\usepackage{amsmath, amssymb, amsthm} %standard AMS packages
\usepackage{marginnote} %marginnotes
\usepackage{gensymb} %miscellaneous symbols
\usepackage{commath} %differential symbols
\usepackage{xcolor} %colours
\usepackage{cancel} %cancelling terms
\usepackage{siunitx} %formatting units
\usepackage{tikz, pgfplots} %diagrams
\usetikzlibrary{calc, hobby, patterns, intersections}
\usepackage{graphicx} %inserting graphics
\usepackage{hyperref} %hyperlinks
\usepackage{datetime} %date and time
\usepackage{ulem} %underline for \emph{}
\usepackage{xfrac} %inline fractions
\usepackage{enumerate, enumitem} %numbered lists
\usepackage{float} %inserting floats
\usepackage{circuitikz} %circuit diagrams
\usepackage[noend]{algpseudocode}
\usepackage{algorithm}

\newcommand\numberthis{\addtocounter{equation}{1}\tag{\theequation}} %adds numbers to specific equations in non-numbered list of equations

\newcommand{\AxisRotator}[1][rotate=0]{
	\tikz [x=0.25cm,y=0.60cm,line width=.2ex,-stealth,#1] \draw (0,0) arc (-150:150:1 and 1);%
} %rotation symbols on axes

\theoremstyle{definition}
\newtheorem{example}{Example}
\newtheorem{definition}{Definition}

\theoremstyle{theorem}
\newtheorem{theorem}{Theorem}

\theoremstyle{remark}
\newtheorem{case}{Case}

\newcommand{\curl}{\mathrm{curl\,}}

\makeatletter
\@addtoreset{section}{part} %resets section numbers in new part
\makeatother

\makeatletter
\@addtoreset{case}{question} %resets section numbers in new part
\makeatother

\renewcommand{\thesubsection}{(\arabic{subsection})}
\renewcommand{\thesection}{(\arabic{section})}

\newcommand{\Not}{{\textsc{ not }}}
\renewcommand{\And}{{\textsc{ and }}}
\newcommand{\Or}{{\textsc{ or }}}
\newcommand{\Xor}{{\textsc{ xor }}}
\newcommand{\Nxor}{{\textsc{ nxor }}}
\newcommand{\Nand}{{\textsc{ nand }}}
\newcommand{\Nor}{{\textsc{ nor }}}

\newcommand{\AND}{\wedge}
\newcommand{\bigAND}{\bigwedge}
\newcommand{\OR}{\vee}
\newcommand{\bigOR}{\bigvee}
\newcommand{\NOT}{\neg}
\newcommand{\XOR}{\oplus}
\newcommand{\bigXOR}{\bigoplus}

\newcommand{\logequiv}{\leftrightarrow}

\newcommand{\dual}{\mathrm{dual\,}}

%section headings on left
\makeatletter
\def\specialsection{\@startsection{section}{1}%
	\z@{\linespacing\@plus\linespacing}{.5\linespacing}%
	%  {\normalfont\centering}}% DELETED
	{\normalfont}}% NEW
\def\section{\@startsection{section}{1}%
	\z@{.7\linespacing\@plus\linespacing}{.5\linespacing}%
	%  {\normalfont\scshape\centering}}% DELETED
	{\normalfont\scshape}}% NEW
\makeatother

%forces newline after subsection
\makeatletter
\def\subsection{\@startsection{subsection}{3}%
	\z@{.5\linespacing\@plus.7\linespacing}{.1\linespacing}%
	{\normalfont\itshape}}
\makeatother

\newcommand{\strangesection}[1]{\renewcommand{\thesection}{#1}\section}
\newcommand{\strangesubsection}[1]{\renewcommand{\thesubsection}{#1}\subsection}

\settasks{counter-format = tsk[1].}

\SetupExSheets{solution/print = true}

%opening
\title{Digital Logic Systems : Assignment 6}
\author{
	Aakash Jog\\
	ID : 989323563\\
	\&\\
	Dustin Chalchinsky\\
	ID : 209741891\\
	\&\\
	Paul Mierau\\
	ID : 932384233
	}
\date{\formatdate{26}{5}{2015}}

\begin{document}
	
\maketitle
%\setlength{\mathindent}{0pt}

\begin{question}
	Prove that every output bit of a decoder depends on all the inputs.
\end{question}

\begin{solution}
	If possible, let an output bit of a decoder, $\textsc{decoder}(n)$, be independent of one of the inputs.\\
	Therefore, WLG, let the $k^{\text{th}}$ bit of the output be independent of the $l^{\text{th}}$ bit of the input.\\
	Let the input be $i[1:n]$ and the output be $o[1:2^n]$.\\
	Therefore, $o[k]$ is unchanged by $flip_{l}(i)$.\\
	Let $o_1$ be the output of $\textsc{decoder}(n)$ for some $i_1$, such that $o_1[k] = 1$.\\
	Therefore, $o_1[k] = 1$ for $flip_l(i_1)$.\\
	Therefore, either $wt(o_1) \neq 1$, or the output of $\textsc{decoder}(n)$ is the same for $i_1$ and $flip_l(i_1)$.\\
	If $wt(o)$, it contradicts the definition of the decoder.\\
	If the outputs for $i_1$ and $flip_l(i_1)$ are the same, it contradicts the fact that any decoder is one-to-one.\\
	Hence, every output bit of $\textsc{decoder}(n)$ must be dependent on all the input bits.
\end{solution}

\begin{question}
	Prove that the logical connective that corresponds to a MUX-gate is complete.
	(Hint: Implement a complete set of connectives using only a MUX and constants.)
\end{question}

\begin{solution}
	Let $D[0]$, $D[1]$, and $S$ be the inputs of a MUX-gate.
	Let $Y$ be the output of the MUX-gate.\\
	Therefore,\\
	\begin{tabular}{|c|c|c||c|}
		\hline
		$D[0]$ & $D[1]$ & $S = X_1$ & $Y$ \\
		\hline
		1      & 0      & 0         & 1   \\
		1      & 0      & 1         & 0   \\
		\hline
	\end{tabular}\\
	Therefore, if $D[0] = 1$, $D[1] = 0$, $S = X_1$, $Y = \NOT X_1$.\\
	\begin{tabular}{|c|c|c||c|}
		\hline
		$D[0]$ & $D[1] = X_1$ & $S = X_2$ & $Y$ \\
		\hline
		0      & 0            & 0         & 0   \\
		0      & 0            & 1         & 0   \\
		0      & 1            & 0         & 0   \\
		0      & 1            & 1         & 1   \\
		\hline
	\end{tabular}\\
	Therefore, if $D[0] = 1$, $D[1] = X_1$, $S = X_2$, $Y = X_1 \AND X_2$.\\
	\begin{tabular}{|c|c|c||c|}
		\hline
		$D[0] = X_1$ & $D[1]$ & $S = X_2$ & $Y$ \\
		\hline
		0            & 1      & 0         & 0   \\
		0            & 1      & 1         & 1   \\
		1            & 1      & 0         & 1   \\
		1            & 1      & 1         & 1   \\
		\hline
	\end{tabular}\\
	Therefore, if $D[0] = X_1$, $D[1] = 1$, $S = X_2$, $Y = X_1 \OR X_2$.\\
	Therefore, as the complete set of connectives $\left\{ \Not, \And, \Or \right\}$ can be constructed using a MUX-gate only, it is a complete set of connectives.
\end{solution}

\begin{question}
	Design a combinational circuit $JO(n)$ (just one) specified as follows:\\
	\begin{tabular}[c]{l l}
		Input         & $x \in \{0,1\}^n$                            \\
		Output        & $y,z \in \{0,1\}$                            \\
		Functionality & $z = 1 \iff x = 0^n$, $y = 1 \iff wt(x) = 1$ \\
	\end{tabular}
	\begin{enumerate}
		\item Prove the correctness of your design.
		\item Analyze the asymptotic cost of your design.
		\item Analyze the asymptotic propagation delay of your design.
		\item Prove asymptotic lower bounds on the cost and propagation delay of $JO(n)$.
	\end{enumerate}
\end{question}

\begin{solution}
	\begin{figure}[H]
		\begin{circuitikz}[scale = 0.5]
			\begin{scope}
				\node [left] at (-5,5) {$x$};

				\draw ($ (-4,5) + (2pt,2pt) $) -- ($ (-4,5) + (-2pt,-2pt) $) node [below] {$n$};

				\draw (-5,5) to [generic = $\Not_n$] (0,5);

				\draw ($ (-1,5) + (2pt,2pt) $) -- ($ (-1,5) + (-2pt,-2pt) $) node [below] {$n$};

				\draw (0,5) -| (0,10);
				\draw (0,5) -| (0,8);
				\draw (0,5) -| (0,6);
				\draw (0,5) -| (0,4);
				\draw (0,5) -| (0,0);
	
				\draw ($ (1,15) + (2pt,2pt) $) -- ($ (1,15) + (-2pt,-2pt) $) node [below] {$n$};
				\draw ($ (1,10) + (2pt,2pt) $) -- ($ (1,10) + (-2pt,-2pt) $) node [below] {$n$};
				\draw ($ (1,8) + (2pt,2pt) $) -- ($ (1,8) + (-2pt,-2pt) $) node [below] {$n$};
				\draw ($ (1,6) + (2pt,2pt) $) -- ($ (1,6) + (-2pt,-2pt) $) node [below] {$n$};
				\draw ($ (1,4) + (2pt,2pt) $) -- ($ (1,4) + (-2pt,-2pt) $) node [below] {$n$};
				\draw ($ (1,0) + (2pt,2pt) $) -- ($ (1,0) + (-2pt,-2pt) $) node [below] {$n$};

				\draw (0,10) to [generic = $flip_0$] (5,10);
				\draw (0,8) to [generic = $flip_1$] (5,8);
				\draw (0,6) to [generic = $flip_2$] (5,6);
				\draw (0,4) to [generic = $flip_3$] (5,4);
				\draw (0,0) to [generic = $flip_n$] (5,0);
	
				\draw (5,10) to [generic = $\And_n$] (10,10);
				\draw (5,8) to [generic = $\And_n$] (10,8);
				\draw (5,6) to [generic = $\And_n$] (10,6);
				\draw (5,4) to [generic = $\And_n$] (10,4);
				\node [above] at (2.5,2) {$\vdots$};
				\draw (5,0) to [generic = $\And_n$] (10,0);
	
				\draw ($ (5,10) + (2pt,2pt) $) -- ($ (5,10) + (-2pt,-2pt) $) node [below] {$n$};
				\draw ($ (5,8) + (2pt,2pt) $) -- ($ (5,8) + (-2pt,-2pt) $) node [below] {$n$};
				\draw ($ (5,6) + (2pt,2pt) $) -- ($ (5,6) + (-2pt,-2pt) $) node [below] {$n$};
				\draw ($ (5,4) + (2pt,2pt) $) -- ($ (5,4) + (-2pt,-2pt) $) node [below] {$n$};
				\draw ($ (5,0) + (2pt,2pt) $) -- ($ (5,0) + (-2pt,-2pt) $) node [below] {$n$};
	
				\draw (10,10) |- (12,5);
				\draw (10,8) |- (12,5);
				\draw (10,6) |- (12,5);
				\draw (10,4) |- (12,5);
				\draw (10,0) |- (12,5);
	
				\draw (12,5) to [generic = $\Or_n$] (17,5);
				\draw ($ (12,5) + (2pt,2pt) $) -- ($ (12,5) + (-2pt,-2pt) $) node [below] {$n$};
	
				\node [right] at (17,5) {$y$};
			\end{scope}

			\begin{scope}
				\draw (-1,5) -| (0,15);

				\draw (0,15) to [generic = $\And_n$] (5,15);

				\draw (5,15) -- (17,15);

				\node [right] at (17,15) {$z$};
			\end{scope}
		\end{circuitikz}
		\caption{$JO(n)$}
	\end{figure}
	Where $\And_n$ is $\And$ of $n$ bits, implemented using an optimized $\And$ tree, $\Or_n$ is $\Or$ of $n$ bits, implemented using an optimized $\Or$.\\
	$\Not_n$ and $flip_i$ are implemented as shown.
	\begin{figure}[H]
		\begin{circuitikz}[scale = 0.6]
			\draw (2,10) node [not port] (not0) {};
			\draw (2,8) node [not port] (not1) {};
			\draw (2,6) node [not port] (not2) {};
			\draw (2,4) node [not port] (not3) {};
			\node at (2,2) {$\vdots$};
			\draw (2,0) node [not port] (not4) {};

			\node [left] at (not0.in) {$x[0]$};
			\node [left] at (not1.in) {$x[1]$};
			\node [left] at (not2.in) {$x[2]$};
			\node [left] at (not3.in) {$x[3]$};
			\node [left] at (not4.in) {$x[n]$};

			\draw (not0.out) -| (4,5);
			\draw (not1.out) -| (4,5);
			\draw (not2.out) -| (4,5);
			\draw (not3.out) -| (4,5);
			\draw (not4.out) -| (4,5);

			\draw (4,5) -- (6,5);

			\draw ($ (5,5) + (2pt,2pt) $) -- ($ (5,5) + (-2pt,-2pt) $) node [below] {$n$};
		\end{circuitikz}
		\caption{$\Not_n$}
	\end{figure}
	\begin{figure}[H]
		\begin{circuitikz}[scale = 0.6]
			\node at (2,6) {$\vdots$};
			\draw (2,4) node [not port] (not) {};
			\node at (2,2) {$\vdots$};

			\node [left] at (1,10) {$x[0]$};
			\node [left] at (1,8) {$x[1]$};
			\node [left] at (1,4) {$x[i]$};
			\node [left] at (not4.in) {$x[n]$};

			\draw (1,10) -| (4,5);
			\draw (1,8) -| (4,5);
			\draw (not.out) -| (4,5);
			\draw (1,0) -| (4,5);

			\draw (4,5) -- (6,5);

			\draw ($ (5,5) + (2pt,2pt) $) -- ($ (5,5) + (-2pt,-2pt) $) node [below] {$n$};
		\end{circuitikz}
		\caption{$flip_i(x)$}
	\end{figure}
	\begin{enumerate}[leftmargin = *]
		\item
			\begin{align*}
				y &= \bigXOR_{i = 0}^{n - 1} \overline{x[0]} \AND \dots \AND x[i] \AND \dots \AND \overline{x[n - 1]}\\
				z &= \bigAND_{i = 0}^{n - 1} \overline{x[i]}
			\end{align*}
			If $n = 1$,\\
			\begin{align*}
				y &= x[0] \XOR \overline{x[1]}\\
				z &= \overline{x[0]} \AND \overline{x[1]}
			\end{align*}
			Therefore,\\
			\begin{tabular}{|c|c||c|c|}
				\hline
				$x[0]$ & $x[1]$ & $y$ & $z$ \\
				\hline
				0      & 0      & 0   & 1   \\
				0      & 1      & 1   & 0   \\
				1      & 0      & 1   & 0   \\
				1      & 1      & 0   & 0   \\
				\hline
			\end{tabular}\\
			Therefore, the circuit $JO(2)$ works as intended.\\
			If possible let $JO(n)$ work as intended.\\
			Therefore,
			\begin{align*}
				y_n &= 1 &\iff&& wt(x_n) &= 1\\
				z_n &= 1 &\iff&& x_n &= 0^n
			\end{align*}
			Therefore,
			\begin{align*}
				y_{n + 1} &= \bigXOR_{i = 0}^{n} \overline{x[0]} \AND \dots \AND x[i] \AND \dots \AND \overline{x[n]}\\
				&= \bigXOR_{i = 0}^{n} \left( \overline{x[0]} \AND \dots \AND x[i] \AND \dots \AND \overline{x[n - 1]} \right) \AND \overline{x[n]}\\
				&= \quad \left( \bigXOR_{i = 0}^{n - 1} \left( \overline{x[0]} \AND \dots \AND x[i] \AND \dots \AND \overline{x[n - 1]} \right) \AND \overline{x[n]} \right)\\
				&\quad \XOR \left( \overline{x[0]} \AND \dots \AND  \overline{x[n - 1]} \AND x[n] \right)\\
				&= y_n \AND \overline{x[n]} \XOR \left( \overline{x[0]} \AND \dots \AND  \overline{x[n - 1]} \AND x[n] \right)
			\end{align*}
			If $x_{n + 1}[n] = 0$,
			\begin{align*}
				y_{n + 1} &= \left( y_n \AND 1 \right) \XOR \left( z_n \AND 0 \right)\\
				&= y_1 \XOR 0\\
				&= y_1\\
				&= wt(x_n)\\
				&= wt(x_{n + 1})
			\end{align*}
			If $x_{n + 1}[n] = 1$,
			\begin{align*}
				y_{n + 1} &= \left( y_n \AND 0 \right) \XOR \left( z_n \AND 1 \right)\\
				&= 0 \XOR z_n
			\end{align*}
			$x_n = 0^n$ if and only if  $z_n = 1$.\\
			If and only if,
			\begin{align*}
				y_{n + 1} &= 0 \XOR 1\\
				&= 1
			\end{align*}
			Therefore, $y_{n + 1} = 1$ if and only if $x[n] = 1$ and $z_n = 1$, i.e. $x[0] = \dots = x[n - 1] = 0$.\\
			Therefore, $wt(x_{n + 1}) = 1$.\\
			$x_n \neq 0^n$ if and only if $z_n = 0$.\\
			If and only if,
			\begin{align*}
				y_{n + 1} &= 0 \XOR 0\\
				&= 0
			\end{align*}
			Therefore, $y_{n + 1} = 0$ if and only if $x[n] = 1$ and $z_n = 0$, i.e. at least one of $x[0], \dots, x[n - 1]$ is 1.
			Therefore, $wt(x_{n + 1}) > 1$.
			~\\
			Therefore, $y_{n + 1} = 1$ if and only if $wt(x_{n + 1}) = 1$.\\
			~\\
			~\\
			\begin{align*}
				z_{n + 1} &= \bigAND_{i = 0}^{n} \overline{x_{n + 1}[i]}\\
				&= \left( \bigAND_{i = 0}^{n - 1} \overline[i] \right) \AND \overline{x_{n + 1}[n]}\\
				&= z_n \AND \overline{x_{n + 1}[n]}
			\end{align*}
			If $x_{n + 1}[n] = 0$,
			\begin{align*}
				z_{n + 1} &= z_n \AND 1\\
				&= z_n
			\end{align*}
			Therefore, $x_{n + 1} = 0^{n + 1}$ if and only if $x_n = 0^n$.\\
			If $x_{n + 1}[n] = 1$,
			\begin{align*}
				z_{n + 1} &= z_n \AND 0\\
				&= 0
			\end{align*}
			Therefore, $x_{n + 1} \neq 0^{n + 1}$ if and only if $x_{n + 1}[n] = 1$.\\
			Hence, by induction, $JO(n)$ works as intended, $\forall n \ge 2$.
			\qed
		\item
			\begin{align*}
				c(y) &= \quad c(\Not_n)\\
				&\quad + n \cdot c(flip)\\
				&\quad + n \cdot c(\And_n)\\
				&\quad + c(\Or_n)\\
				&=\quad n \cdot c(\Not)\\
				&\quad + n (n - 1) c(\Not)\\
				&\quad + n (n - 1) c(\And)\\
				&\quad + (n - 1) c(\Or)
			\end{align*}
			\begin{align*}
				\text{cost}(z) &= (n - 1) \text{cost}(\Or) + \text{cost}(\Not)
			\end{align*}
			Therefore,
			\begin{align*}
				c(y) &= O\left( n^2 \right)\\
				c(z) &= O (n)\\
				\therefore \text{cost} &= O\left( n^2 \right)
			\end{align*}
		\item
			\begin{align*}
				d(y) &= d(\Not_n) + d(flip_i) + d(\And_n) + d(\Or_n)\\
				d(z) &= d(\Not_n) + d(\And_n)
			\end{align*}
			Therefore,
			\begin{align*}
				\text{delay} &= \max\left\{ d(y) , d(z) \right\}\\
				&= d(\Not_n) + d(flip_i) + d(\And_n) + d(\Or_n)\\
				&= \Theta(\log_2 n)
			\end{align*}
	\end{enumerate}
\end{solution}

\end{document}
