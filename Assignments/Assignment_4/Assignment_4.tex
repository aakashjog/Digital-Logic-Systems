\documentclass[fleqn, a4paper, 12pt, oneside]{amsart}
\usepackage{exsheets, tasks}
\usepackage{amsmath, amssymb, amsthm} %standard AMS packages
\usepackage{marginnote} %marginnotes
\usepackage{gensymb} %miscellaneous symbols
\usepackage{commath} %differential symbols
\usepackage{xcolor} %colours
\usepackage{cancel} %cancelling terms
\usepackage{siunitx} %formatting units
\usepackage{tikz, pgfplots} %diagrams
\usetikzlibrary{calc, hobby, patterns, intersections}
\usepackage{graphicx} %inserting graphics
\usepackage{hyperref} %hyperlinks
\usepackage{datetime} %date and time
\usepackage{ulem} %underline for \emph{}
\usepackage{xfrac} %inline fractions
\usepackage{enumerate} %numbered lists
\usepackage{float} %inserting floats
\usepackage{circuitikz} %circuit diagrams
\usepackage[noend]{algpseudocode}
\usepackage{algorithm}

\newcommand\numberthis{\addtocounter{equation}{1}\tag{\theequation}} %adds numbers to specific equations in non-numbered list of equations

\newcommand{\AxisRotator}[1][rotate=0]{
	\tikz [x=0.25cm,y=0.60cm,line width=.2ex,-stealth,#1] \draw (0,0) arc (-150:150:1 and 1);%
} %rotation symbols on axes

\theoremstyle{definition}
\newtheorem{example}{Example}
\newtheorem{definition}{Definition}

\theoremstyle{theorem}
\newtheorem{theorem}{Theorem}

\theoremstyle{remark}
\newtheorem{case}{Case}

\newcommand{\curl}{\mathrm{curl\,}}

\makeatletter
\@addtoreset{section}{part} %resets section numbers in new part
\makeatother

\makeatletter
\@addtoreset{case}{question} %resets section numbers in new part
\makeatother

\renewcommand{\thesubsection}{(\arabic{subsection})}
\renewcommand{\thesection}{(\arabic{section})}

\newcommand{\Not}{{\textsc{ not }}}
\renewcommand{\And}{{\textsc{ and }}}
\newcommand{\Or}{{\textsc{ or }}}
\newcommand{\Xor}{{\textsc{ xor }}}
\newcommand{\Nxor}{{\textsc{ nxor }}}
\newcommand{\Nand}{{\textsc{ nand }}}
\newcommand{\Nor}{{\textsc{ nor }}}

\newcommand{\AND}{\wedge}
\newcommand{\OR}{\vee}
\newcommand{\NOT}{\neg}
\newcommand{\XOR}{\oplus}

\newcommand{\logequiv}{\leftrightarrow}

\newcommand{\dual}{\mathrm{dual\,}}

%section headings on left
\makeatletter
\def\specialsection{\@startsection{section}{1}%
	\z@{\linespacing\@plus\linespacing}{.5\linespacing}%
	%  {\normalfont\centering}}% DELETED
	{\normalfont}}% NEW
\def\section{\@startsection{section}{1}%
	\z@{.7\linespacing\@plus\linespacing}{.5\linespacing}%
	%  {\normalfont\scshape\centering}}% DELETED
	{\normalfont\scshape}}% NEW
\makeatother

%forces newline after subsection
\makeatletter
\def\subsection{\@startsection{subsection}{3}%
	\z@{.5\linespacing\@plus.7\linespacing}{.1\linespacing}%
	{\normalfont\itshape}}
\makeatother

\newcommand{\strangesection}[1]{\renewcommand{\thesection}{#1}\section}
\newcommand{\strangesubsection}[1]{\renewcommand{\thesubsection}{#1}\subsection}

\settasks{counter-format = tsk[1].}

\SetupExSheets{solution/print = true}

%opening
\title{Digital Logic Systems : Assignment 4}
\author{
	Aakash Jog\\
	ID : 989323563\\
	\&\\
	Dustin Chalchinsky\\
	ID : 209741891\\
	\&\\
	Paul Mierau\\
	ID : 932384233
	}
\date{\formatdate{21}{4}{2015}}

\begin{document}
	
\maketitle
%\setlength{\mathindent}{0pt}

\begin{question}
	Let $p$, $q$, $r$ denote boolean formulas.
	Prove that if $p$ is logically equivalent to $q$, and $q$ is logically equivalent to $r$, then $p$ is logically equivalent to $r$.
\end{question}

\begin{solution}
	\begin{tabular}[c]{|c|c|c||c|c|c|c|}
		\hline
		$p$ & $q$ & $r$ & $p \logequiv q$ & $q \logequiv r$ & $(p \logequiv q) \AND (q \logequiv r)$ & $p \logequiv r$\\
		\hline
		0   & 0   & 0   & 1               & 1               & 1                                      & 1 \\
		0   & 0   & 1   & 1               & 0               & 0                                      & 0 \\
		0   & 1   & 0   & 0               & 0               & 1                                      & 1 \\
		0   & 1   & 1   & 0               & 1               & 0                                      & 0 \\
		1   & 0   & 0   & 0               & 1               & 0                                      & 0 \\
		1   & 0   & 1   & 0               & 0               & 1                                      & 1 \\
		1   & 1   & 0   & 1               & 0               & 0                                      & 0 \\
		1   & 1   & 1   & 1               & 1               & 1                                      & 1 \\
		\hline
	\end{tabular}\\
	Therefore, as the columns of ($(p \logequiv q) \AND (q \logequiv r)$) and ($p \logequiv r$) are identical, they are equivalent.
	Hence, if  $p$ is logically equivalent to $q$, and $q$ is logically equivalent to $r$, then $p$ is logically equivalent to $r$.
\end{solution}

\begin{question}
	Let $L(\varphi)$ denote the number of vertices in the parse tree of $\varphi$ that are not labelled by negation.
	Prove that $L \left( \mathrm{DM}(\varphi) \right) = L(\varphi)$, for every boolean formula $\varphi$ that uses the logical connectives $\{\Not, \Or, \And\}$.	
\end{question}

\begin{solution}
	In the De Morgan dual of $\varphi$, all vertices labeled by $\Or$ are changed to vertices labeled by $\And$, and vice-versa.
	Depending on $\varphi$, vertices labelled by $\Not$ are be added or removed.\\
	Therefore, the number of vertices labelled by $\And$ in $\varphi$ is equal to the number of vertices labelled by $\Or$ in $\mathrm{DM}(\varphi)$.
	Similarly for $\Or$ in $\varphi$ and $\And$ in $\mathrm{DM}(\varphi)$.\\
	The number of vertices labelled by variables is unchanged in $\varphi$ and $\mathrm{DM}(\varphi)$.\\
	Therefore, the total number of vertices not labelled by negation, i.e. labelled only be variables, $\And$, and $\Or$ is the same in the parse trees of $\varphi$ and $\mathrm{DM}(\varphi)$.
	Therefore, $L \left( \mathrm{DM}(\varphi) \right) = L(\varphi)$.
	\qed
\end{solution}

\addtocounter{question}{1}

%\begin{question}
%	For two vectors $x = (x_0, \dots, x_{k - 1})$ and $y = (y_0, \dots, y_{k - 1})$ we say that $x \geq y$ if $x_i \geq y_i$ for every $i \in \{0, 1,\dots, k - 1\}$.
%	A function $f : \{0,1\}^k \rightarrow \{0,1\}$ is monotone if $\forall x,y \in k$
%	\begin{equation*}
%		x \geq y \implies f(x) \geq f(y)
%	\end{equation*}
%	Prove that if $\varphi$ uses only connectives in $\{\And, \Or\}$, then the boolean function $B_\varphi$ is monotone.
%\end{question}
%
%\begin{solution}
%
%\end{solution}<++>

\begin{question}
	Add the following two reduction rules to Algorithm $\mathrm{DM}(\varphi)$ so that you can also deal with the $\Xor$ and $\Nxor$ connectives:
	\begin{enumerate}
		\item If $\varphi = (\varphi_1 \Xor \varphi_2)$, then return $(\mathrm{DM}(\varphi_1) \Nxor \mathrm{DM}(\varphi_2))$.
		\item If $\varphi = (\varphi_1 \Nxor \varphi_2)$, then return $(\mathrm{DM}(\varphi_1) \Xor \mathrm{DM}(\varphi_2))$.
	\end{enumerate}
	Prove that, even after this modification, $\mathrm{DM}(\varphi) \logequiv \neg \varphi$ is a tautology.
\end{question}

\begin{solution}
	\begin{align*}
		x \Xor y &= (x \OR y) \AND (x \AND (\NOT y))\\
		x \Nxor y &= \NOT ((x \OR y) \AND (x \AND (\NOT y)))
	\end{align*}
	Therefore, if
	\begin{align*}
		\varphi &\equiv \varphi_1 \Xor \varphi_2\\
		\therefore \varphi &\equiv (\varphi_1 \OR \varphi_2) \AND (\varphi_1 \AND \NOT \varphi_2)\\
		\therefore \mathrm{DM}(\varphi) &\equiv \mathrm{DM}((\varphi_1 \OR \varphi_2) \AND (\varphi_1 \AND \NOT \varphi_2))\\
		&\equiv (\NOT \varphi_1 \AND \NOT \varphi_2) \OR (\NOT \varphi_1 \OR \varphi_2)\\
		&\equiv \NOT (\varphi_1 \OR \varphi_2) \OR \NOT (\varphi_1 \AND \NOT \varphi_2)\\
		&\equiv \NOT ((\varphi_1 \OR \varphi_2) \AND (\varphi_1 \AND \NOT \varphi_2))\\
		&\equiv \NOT (\varphi_1 \Xor \varphi_2)\\
		&\equiv \NOT (\varphi)
	\end{align*}
	Therefore, as $\mathrm{DM}(\varphi) \equiv \NOT \varphi$, $\mathrm{DM}(\varphi) \logequiv \NOT \varphi$ is a tautology.\\
	~\\
	Similarly, if
	\begin{align*}
		\varphi &\equiv \varphi_1 \Nxor \varphi_2\\
		\therefore \varphi &\equiv \NOT ((\varphi_1 \OR \varphi_2) \AND (\varphi_1 \AND \NOT \varphi_2))\\
		\therefore \mathrm{DM}(\varphi) &\equiv \mathrm{DM}(\NOT ((\varphi_1 \OR \varphi_2) \AND (\varphi_1 \AND \NOT \varphi_2)))\\
		&\equiv \mathrm{DM} (\NOT (\varphi_1 \OR \varphi_2) \OR \NOT (\varphi_1 \AND \NOT \varphi_2))\\
		&\equiv \mathrm{DM} ((\NOT \varphi_1 \AND \NOT \varphi_2) \OR (\NOT \varphi_1 \OR \varphi_2))\\
		&\equiv (\varphi_1 \OR \varphi_2) \AND (\varphi_1 \AND \NOT \varphi_2)\\
		&\equiv \varphi_1 \Xor \varphi_2\\
		&\equiv \NOT (\varphi_1 \Nxor \varphi_2)\\
		&\equiv \NOT (\varphi)
	\end{align*}
	Therefore, as $\mathrm{DM}(\varphi) \equiv \NOT \varphi$, $\mathrm{DM}(\varphi) \logequiv \NOT \varphi$ is a tautology.
	\qed
\end{solution}

\begin{question}
	Let $\varphi_k$ denote the boolean formula in which the variable $X$ is negated $k$ times. Run algorithm $\mathrm{NNF}(\varphi_k)$.
	What is the outcome?
	Prove your result.
	Hint: distinguish between an even $k$ and an odd $k$.
\end{question}

\begin{solution}
	\begin{align*}
		\varphi_k &= \underbrace{\NOT(\NOT(\dots(\NOT X)))}_{k \text{ times }}\\
		\therefore \mathrm{NNF}(\varphi_k) &= \mathrm{DM}(\mathrm{NNF}(\underbrace{\NOT(\NOT(\dots(\NOT X)))}_{k - 1 \text{ times }}))\\
		&= \underbrace{\mathrm{DM}(\mathrm{DM}(\dots(\mathrm{DM}}_{k - 1 \text{ times }}(\mathrm{NNF}(X)))))\\
		&= \underbrace{\mathrm{DM}(\mathrm{DM}(\dots(\mathrm{DM}}_{k - 1 \text{ times }}(\NOT X))))
	\end{align*}
	\begin{case}[$k$ is odd]
		If $k$ is odd, $k - 1$ must be even.\\
		Therefore, as $\mathrm{DM}(\mathrm{DM}(\varphi))$ is logically equivalent to $\varphi$, the $\dfrac{k - 1}{2}$ pairs of $\mathrm{DM}$ will be cancelled out.
		Hence, $\mathrm{NNF}(\varphi_k)$ is logically equivalent to $\NOT k$.\\
		Therefore,
		\begin{equation*}
			\mathrm{NNF}(\varphi_k) \equiv \NOT k
		\end{equation*}
	\end{case}
	\begin{case}[$k$ is even]
		If $k$ is odd, $k - 1$ must be odd.\\
		Therefore, as $\mathrm{DM}(\mathrm{DM}(\varphi))$ is logically equivalent to $\varphi$, the $\dfrac{k - 2}{2}$ pairs of $\mathrm{DM}$ will be cancelled out.
		Hence, $\mathrm{NNF}(\varphi_k)$ is logically equivalent to $\mathrm{DM}(\NOT k)$.\\
		Therefore,
		\begin{equation*}
			\mathrm{NNF}(\varphi_k) \equiv k
		\end{equation*}
	\end{case}
\end{solution}

\end{document}
