\documentclass[fleqn, a4paper, 12pt, oneside]{amsart}
\usepackage{exsheets, tasks}
\usepackage{amsmath, amssymb, amsthm} %standard AMS packages
\usepackage{marginnote} %marginnotes
\usepackage{gensymb} %miscellaneous symbols
\usepackage{commath} %differential symbols
\usepackage{xcolor} %colours
\usepackage{cancel} %cancelling terms
\usepackage{siunitx} %formatting units
\usepackage{tikz, pgfplots} %diagrams
\usetikzlibrary{calc, hobby, patterns, intersections}
\usepackage{graphicx} %inserting graphics
\usepackage{hyperref} %hyperlinks
\usepackage{datetime} %date and time
\usepackage{ulem} %underline for \emph{}
\usepackage{xfrac} %inline fractions
\usepackage{enumerate} %numbered lists
\usepackage{float} %inserting floats
\usepackage{circuitikz} %circuit diagrams
\usepackage{algorithm, algorithmic}

\newcommand\numberthis{\addtocounter{equation}{1}\tag{\theequation}} %adds numbers to specific equations in non-numbered list of equations

\newcommand{\AxisRotator}[1][rotate=0]{
	\tikz [x=0.25cm,y=0.60cm,line width=.2ex,-stealth,#1] \draw (0,0) arc (-150:150:1 and 1);%
} %rotation symbols on axes

\theoremstyle{definition}
\newtheorem{example}{Example}
\newtheorem{definition}{Definition}

\theoremstyle{theorem}
\newtheorem{theorem}{Theorem}

\newcommand{\curl}{\mathrm{curl\,}}

\makeatletter
\@addtoreset{section}{part} %resets section numbers in new part
\makeatother

\renewcommand{\thesubsection}{(\arabic{subsection})}
\renewcommand{\thesection}{(\arabic{section})}

%\newcommand{\Not}{{\textsc{not}}}
%\renewcommand{\And}{{\textsc{and}}}
%\newcommand{\Or}{{\textsc{or}}}
%\newcommand{\Xor}{{\textsc{xor}}}
%\newcommand{\Nand}{{\textsc{nand}}}
%\newcommand{\Nor}{{\textsc{nor}}}

%\newcommand{\AND}{\wedge}
%\newcommand{\OR}{\vee}

%section headings on left
\makeatletter
\def\specialsection{\@startsection{section}{1}%
	\z@{\linespacing\@plus\linespacing}{.5\linespacing}%
	%  {\normalfont\centering}}% DELETED
	{\normalfont}}% NEW
\def\section{\@startsection{section}{1}%
	\z@{.7\linespacing\@plus\linespacing}{.5\linespacing}%
	%  {\normalfont\scshape\centering}}% DELETED
	{\normalfont\scshape}}% NEW
\makeatother

%forces newline after subsection
\makeatletter
\def\subsection{\@startsection{subsection}{3}%
	\z@{.5\linespacing\@plus.7\linespacing}{.1\linespacing}%
	{\normalfont\itshape}}
\makeatother

\newcommand{\strangesection}[1]{\renewcommand{\thesection}{#1}\section}
\newcommand{\strangesubsection}[1]{\renewcommand{\thesubsection}{#1}\subsection}

\settasks{counter-format = tsk[1].}

\SetupExSheets{solution/print = true}

%opening
\title{Digital Logic Systems : Assignment 1}
\author{
	Aakash Jog\\
	ID : 989323563\\
	\&\\
	Dustin Chalchinsky\\
	ID : 209741891
	}
\date{\formatdate{17}{3}{2015}}

\begin{document}
	
\maketitle
%\setlength{\mathindent}{0pt}

\begin{question}
	Prove that if the vertices of a directed graph G admit a topological ordering (i.e. $(u, v) \in E$ implies that  $\pi(u) < \pi(v)$), then $G$ is acyclic.
\end{question}

\begin{solution}
	If possible, let $G = (V, E)$ be cyclic.\\
	Therefore, after a finite number of runs of the algorithm $TS(V,E)$, there will be a case where there are no sinks.
	In such a case, the algorithm fails.
	Therefore, the vertices of $G$ do not assume a topological sorting.\\
	This contradicts the given condition.\\
	Hence, $G$ must be acyclic.
	\qed
\end{solution}

\begin{question}
	Suggest an algorithm that is input a directed graph and outputs whether the graph is acyclic.
\end{question}

\begin{algorithm}
	\caption{An algorithm that is input a directed graph and outputs whether the graph is acyclic.}
	\begin{algorithmic}
		\IF{$|V| = 0$}
			\STATE let $v \in V$ \AND \RETURN $\pi(v) = 0$ 
			\AND 
			\RETURN acyclic
		\ELSIF{$\exists v \in V$, such that $\deg_{\textnormal{out}} = 0$}
			\RETURN $\left( \textnormal{TS}(V,E) \left( V \setminus v, E \setminus E_v \right) \right)$ extended by $\pi(v) = |V| - 1$.
		\ELSE
			\RETURN cyclic
		\ENDIF
	\end{algorithmic}
\end{algorithm}

\end{document}