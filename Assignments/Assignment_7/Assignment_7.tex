\documentclass[fleqn, a4paper, 11pt, oneside]{amsart}
\usepackage{exsheets, tasks}
\usepackage{amsmath, amssymb, amsthm} %standard AMS packages
\usepackage{marginnote} %marginnotes
\usepackage{gensymb} %miscellaneous symbols
%\usepackage{commath} %differential symbols
\usepackage{xcolor} %colours
\usepackage{cancel} %cancelling terms
\usepackage{siunitx} %formatting units
\usepackage{tikz, pgfplots} %diagrams
\usetikzlibrary{calc, hobby, patterns, intersections}
\usepackage{graphicx} %inserting graphics
\usepackage{hyperref} %hyperlinks
\usepackage{datetime} %date and time
\usepackage{ulem} %underline for \emph{}
\usepackage{xfrac} %inline fractions
\usepackage{enumerate, enumitem} %numbered lists
\usepackage{float} %inserting floats
\usepackage{circuitikz} %circuit diagrams
\usepackage[noend]{algpseudocode}
\usepackage{algorithm}

\newcommand\numberthis{\addtocounter{equation}{1}\tag{\theequation}} %adds numbers to specific equations in non-numbered list of equations

\newcommand{\AxisRotator}[1][rotate=0]{
	\tikz [x=0.25cm,y=0.60cm,line width=.2ex,-stealth,#1] \draw (0,0) arc (-150:150:1 and 1);%
} %rotation symbols on axes

\theoremstyle{definition}
\newtheorem{example}{Example}
\newtheorem{definition}{Definition}

\theoremstyle{theorem}
\newtheorem{theorem}{Theorem}

\theoremstyle{remark}
\newtheorem{case}{Case}

\newcommand{\curl}{\mathrm{curl\,}}

\makeatletter
\@addtoreset{section}{part} %resets section numbers in new part
\makeatother

\makeatletter
\@addtoreset{case}{question} %resets section numbers in new part
\makeatother

\renewcommand{\thesubsection}{(\arabic{subsection})}
\renewcommand{\thesection}{(\arabic{section})}

\newcommand{\Not}{{\textsc{ not }}}
\renewcommand{\And}{{\textsc{ and }}}
\newcommand{\Or}{{\textsc{ or }}}
\newcommand{\Xor}{{\textsc{ xor }}}
\newcommand{\Nxor}{{\textsc{ nxor }}}
\newcommand{\Nand}{{\textsc{ nand }}}
\newcommand{\Nor}{{\textsc{ nor }}}

\newcommand{\AND}{\wedge}
\newcommand{\bigAND}{\bigwedge}
\newcommand{\OR}{\vee}
\newcommand{\bigOR}{\bigvee}
\newcommand{\NOT}{\neg}
\newcommand{\XOR}{\oplus}
\newcommand{\bigXOR}{\bigoplus}

\newcommand{\logequiv}{\leftrightarrow}

\newcommand{\dual}{\mathrm{dual\,}}

%section headings on left
\makeatletter
\def\specialsection{\@startsection{section}{1}%
	\z@{\linespacing\@plus\linespacing}{.5\linespacing}%
	%  {\normalfont\centering}}% DELETED
	{\normalfont}}% NEW
\def\section{\@startsection{section}{1}%
	\z@{.7\linespacing\@plus\linespacing}{.5\linespacing}%
	%  {\normalfont\scshape\centering}}% DELETED
	{\normalfont\scshape}}% NEW
\makeatother

%forces newline after subsection
\makeatletter
\def\subsection{\@startsection{subsection}{3}%
	\z@{.5\linespacing\@plus.7\linespacing}{.1\linespacing}%
	{\normalfont\itshape}}
\makeatother

\newcommand{\strangesection}[1]{\renewcommand{\thesection}{#1}\section}
\newcommand{\strangesubsection}[1]{\renewcommand{\thesubsection}{#1}\subsection}

\settasks{counter-format = tsk[1].}

\SetupExSheets{solution/print = true}

%opening
\title{Digital Logic Systems : Assignment 7}
\author{
	Aakash Jog\\
	ID : 989323563\\
	\&\\
	Dustin Chalchinsky\\
	ID : 209741891\\
	\&\\
	Paul Mierau\\
	ID : 932384233
	}
\date{\formatdate{7}{6}{2015}}

\begin{document}
	
\maketitle
%\setlength{\mathindent}{0pt}

\begin{question}
	Design a combinational circuit that implements the following specification.
	\begin{tabular}{l l}
		        Input & $X[2^k - 1 : 0] \in \{0,1\}^{2^k}$                                                             \\
		       Output & $L, R \in \{0,1\}^k$                                                                           \\
		Functionality & $wt(X) = 2 \implies X\left[ \langle L \rangle \right] = X\left[ \langle R \rangle \right] = 1$ \\
	\end{tabular}
\end{question}

\def\demux#1#2{
	\begin{scope}[shift = {#1}, rotate = {#2}]
		\draw (-0.5,0.5) -- (-0.5,-0.5) -- (0.5,-1) -- (0.5,1) -- cycle;

		\draw (-0.5,0) -- (-1,0);

		\draw (0.5,0.5) -- (1,0.5);
		\draw (0.5,-0.5) -- (1,-0.5);

		\draw (0,-0.75) -- (0,-1);

		\node [left] at (0.5,0.5) {\tiny{$0$}};
		\node [left] at (0.5,-0.5) {\tiny{$1$}};
	\end{scope}
}

\def\JO#1#2#3{
	\begin{scope}[shift = {#1}, rotate = {#2}]
		\draw (-1,-1) rectangle (1,1);

		\draw (-1,0) -- (-2,0);

		\draw (1,0.5) -- (2,0.5);
		\draw (1,-0.5) -- (2,-0.5);

		\node [left] at (1,0.5) {\tiny{$y$}};
		\node [left] at (1,-0.5) {\tiny{$z$}};
		\node at (0,0) {$JO(${#3}$)$};
	\end{scope}
}

\begin{solution}
	\begin{figure}[H]
		\begin{tikzpicture}[scale = 0.93]
			\begin{scope}
				\demux{(5,0)}{0};

				\draw (0,0) -- (4,0);
			
				\draw (6,0.5) -- (6,1) -- (7,1);
				\draw (6,-0.5) -- (6,-1) -- (7,-1);

				\node [left] at (0,0) {$X[2^k - 1 : 2^{k - 1} - 1]$};

				\draw (1,0) -- (1,-4) -- (2,-4);

				\JO{(3,-4)}{0}{$2^{k - 1}$};

				\draw (5,-3.5) -- (5,-1);

				\draw (7,0.5) rectangle (10,1.5);
				\node at (8.5,1) {$f\left( 2^{k - 1} \right)$};

				\draw (7,-0.5) rectangle (10,-1.5);
				\node at (8.5,-1) {$\textsc{decoder}\left( 2^{k - 1} \right)$};

				\draw (10,1) -- (11,1);
				\draw (10,-1) -- (11,-1);
			\end{scope}
			\begin{scope}[shift = {(0,8)}]
				\demux{(5,0)}{0};

				\draw (0,0) -- (4,0);
			
				\draw (6,0.5) -- (6,1) -- (7,1);
				\draw (6,-0.5) -- (6,-1) -- (7,-1);

				\node [left] at (0,0) {$X[2^{k - 1} - 1 : 0]$};

				\draw (1,0) -- (1,-4) -- (2,-4);

				\JO{(3,-4)}{0}{$2^{k - 1}$};

				\draw (5,-3.5) -- (5,-1);

				\draw (7,0.5) rectangle (10,1.5);
				\node at (8.5,1) {$f\left( 2^{k - 1} \right)$};

				\draw (7,-0.5) rectangle (10,-1.5);
				\node at (8.5,-1) {$\textsc{decoder}\left( 2^{k - 1} \right)$};

				\draw (10,1) -- (11,1);
				\draw (10,-1) -- (11,-1);
			\end{scope}
		\end{tikzpicture}
		\caption{$f(2^k)$}
	\end{figure}
\end{solution}

\end{document}
