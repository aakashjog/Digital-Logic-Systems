\documentclass[fleqn, a4paper, 12pt, oneside]{amsart}
\usepackage{exsheets, tasks}
\usepackage{amsmath, amssymb, amsthm} %standard AMS packages
\usepackage{marginnote} %marginnotes
\usepackage{gensymb} %miscellaneous symbols
\usepackage{commath} %differential symbols
\usepackage{xcolor} %colours
\usepackage{cancel} %cancelling terms
\usepackage{siunitx} %formatting units
\usepackage{tikz, pgfplots} %diagrams
\usetikzlibrary{calc, hobby, patterns, intersections}
\usepackage{graphicx} %inserting graphics
\usepackage{hyperref} %hyperlinks
\usepackage{datetime} %date and time
\usepackage{ulem} %underline for \emph{}
\usepackage{xfrac} %inline fractions
\usepackage{enumerate} %numbered lists
\usepackage{float} %inserting floats
\usepackage{circuitikz} %circuit diagrams
\usepackage[noend]{algpseudocode}
\usepackage{algorithm}

\newcommand\numberthis{\addtocounter{equation}{1}\tag{\theequation}} %adds numbers to specific equations in non-numbered list of equations

\newcommand{\AxisRotator}[1][rotate=0]{
	\tikz [x=0.25cm,y=0.60cm,line width=.2ex,-stealth,#1] \draw (0,0) arc (-150:150:1 and 1);%
} %rotation symbols on axes

\theoremstyle{definition}
\newtheorem{example}{Example}
\newtheorem{definition}{Definition}

\theoremstyle{theorem}
\newtheorem{theorem}{Theorem}

\newcommand{\curl}{\mathrm{curl\,}}

\makeatletter
\@addtoreset{section}{part} %resets section numbers in new part
\makeatother

\renewcommand{\thesubsection}{(\arabic{subsection})}
\renewcommand{\thesection}{(\arabic{section})}

\newcommand{\Not}{{\textsc{not}}}
\renewcommand{\And}{{\textsc{and}}}
\newcommand{\Or}{{\textsc{or}}}
\newcommand{\Xor}{{\textsc{xor}}}
\newcommand{\Nand}{{\textsc{nand}}}
\newcommand{\Nor}{{\textsc{nor}}}

\newcommand{\AND}{\wedge}
\newcommand{\OR}{\vee}

%section headings on left
\makeatletter
\def\specialsection{\@startsection{section}{1}%
	\z@{\linespacing\@plus\linespacing}{.5\linespacing}%
	%  {\normalfont\centering}}% DELETED
	{\normalfont}}% NEW
\def\section{\@startsection{section}{1}%
	\z@{.7\linespacing\@plus\linespacing}{.5\linespacing}%
	%  {\normalfont\scshape\centering}}% DELETED
	{\normalfont\scshape}}% NEW
\makeatother

%forces newline after subsection
\makeatletter
\def\subsection{\@startsection{subsection}{3}%
	\z@{.5\linespacing\@plus.7\linespacing}{.1\linespacing}%
	{\normalfont\itshape}}
\makeatother

\newcommand{\strangesection}[1]{\renewcommand{\thesection}{#1}\section}
\newcommand{\strangesubsection}[1]{\renewcommand{\thesubsection}{#1}\subsection}

\settasks{counter-format = tsk[1].}

\SetupExSheets{solution/print = true}

%opening
\title{Digital Logic Systems : Assignment 3}
\author{
	Aakash Jog\\
	ID : 989323563\\
	\&\\
	Dustin Chalchinsky\\
	ID : 209741891
	}
\date{\formatdate{14}{4}{2015}}

\begin{document}
	
\maketitle
%\setlength{\mathindent}{0pt}

\begin{question}
	Suggest an algorithm for computing the binary representation of a number $x \in [0, 2^k - 1]$ using $k$-bit strings. Your algorithm should compute the representation from the LSB to the MSB.
\end{question}

\begin{solution}
	\begin{algorithm}
		\caption{An algorithm for computing the binary representation of a number using $k$-bit strings.}
		\begin{algorithmic}[1]
			\State Let $i = 0$
			\State Let $B[(k - 1) : 0] = 0$
			\While{$i < k$}
				\State $B[i] = (x \% 2)$
				\State $x = \left\lfloor \dfrac{x}{2} \right\rfloor$
				\State $i++$
			\EndWhile
			\State \Return $B[(k - 1) : 0]$
		\end{algorithmic}
	\end{algorithm}
\end{solution}

\begin{question}
	This question deals with the conversion of a hexadecimal string to a binary string such that both strings represent the same natural number. Let $H[k - 1 : 0]$ denote a $k$-digit hexadecimal string. Let $X_H[n - 1 : 0]$ denote an $n$-bit binary string. Answer the following questions.
	\begin{tasks}
		\task 
			Define the conversion, i.e., define the binary string $X_H$ as a function of the hexadecimal string $H$.
		\task 
			Let $h$ denote the number represented by the hexadecimal string $H$.
			Prove that
			\begin{equation*}
				\langle X_H \rangle = h
			\end{equation*}
	\end{tasks}
\end{question}

\begin{solution}
	\begin{tasks}
		\task
			Let $\mathrm{hex-to-bin}(h)$ be a function defined as
			\begin{align*}
				\mathrm{hex-to-bin}(h) &=
					\begin{cases}
						0000 &;\quad h = 0\\
						0001 &;\quad h = 1\\
						0010 &;\quad h = 2\\
						0011 &;\quad h = 3\\
						0100 &;\quad h = 4\\
						0101 &;\quad h = 5\\
						0110 &;\quad h = 6\\
						0111 &;\quad h = 7\\
						1000 &;\quad h = 8\\
						1001 &;\quad h = 9\\
						1010 &;\quad h = A\\
						1011 &;\quad h = B\\
						1100 &;\quad h = C\\
						1101 &;\quad h = D\\
						1110 &;\quad h = E\\
						1111 &;\quad h = F\\
					\end{cases}
			\end{align*}
			Assuming that $X_H$ and $H$ represent the same number,
			\begin{align*}
				X_H \left[ (n - 1) : (n - 4) \right] = \mathrm{hex-to-bin} \left( H[k - 1] \right)
			\end{align*}
		\task
			$\langle X_H \rangle$ is the number represented by $X_H$.
			By definition, $h$ is the number represented by $H$.\\
			Therefore, by the initial assumptions, $X_H$ and $H$ represent the same number.
	\end{tasks}
\end{solution}

%\begin{question}
%	Let $s(m)$ denote the number of parse trees with $m$ nodes with the variables $U = \{X_1, \dots, X_n\}$ and the logical connective \{\And, \Or, \Not\}.
%	Let $f(n)$ denote the number of boolean functions with domain $\{0,1\}^n$ and range $\{0,1\}$. How big does $m$ have to be so that $s(m) > f(n)$? (For example: is $m = n^2$ big enough? How about $m = 2^n$?) What does this imply on the size of the parse trees needed to represent all the Boolean functions over $n$ variables?
%\end{question}

\end{document}