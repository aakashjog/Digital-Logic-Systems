\documentclass[fleqn, a4paper, 12pt, twoside]{article}
\usepackage{exsheets}
\usepackage{amsmath, amssymb, amsthm} %standard AMS packages
\usepackage{marginnote} %marginnotes
\usepackage{gensymb} %miscellaneous symbols
\usepackage{commath} %differential symbols
\usepackage{xcolor} %colours
\usepackage{cancel} %cancelling terms
\usepackage{siunitx} %formatting units
\usepackage{tikz, pgfplots} %diagrams
	\usetikzlibrary{calc, hobby, patterns, intersections}
\usepackage{graphicx} %inserting graphics
\usepackage{hyperref} %hyperlinks
\usepackage{datetime} %date and time
\usepackage{ulem} %underline for \emph{}
\usepackage{xfrac} %inline fractions
\usepackage{enumerate} %numbered lists
\usepackage{float} %inserting floats
\usepackage{circuitikz} %circuit diagrams

\newcommand\numberthis{\addtocounter{equation}{1}\tag{\theequation}} %adds numbers to specific equations in non-numbered list of equations

\newcommand{\AxisRotator}[1][rotate=0]{
	\tikz [x=0.25cm,y=0.60cm,line width=.2ex,-stealth,#1] \draw (0,0) arc (-150:150:1 and 1);%
} %rotation symbols on axes

\theoremstyle{definition}
\newtheorem{example}{Example}
\newtheorem{definition}{Definition}

\theoremstyle{theorem}
\newtheorem{theorem}{Theorem}

\newcommand{\curl}{\mathrm{curl\,}}

\newcommand{\NOT}{{\textsc{not}}}
\newcommand{\AND}{{\textsc{and}}}
\newcommand{\OR}{{\textsc{or}}}
\newcommand{\XOR}{{\textsc{xor}}}

\makeatletter
\@addtoreset{section}{part} %resets section numbers in new part
\makeatother

%opening
\title{Digital Logic Systems}
\author{Aakash Jog}
\date{2014-15}

\begin{document}

\maketitle
%\setlength{\mathindent}{0pt}

\tableofcontents

\newpage
\section{Lecturer Information}

\textbf{Prof. Guy Even}\\
~\\
Office: Wolfson 202\\
Telephone: 03-640-7769\\
E-mail: guy@eng.tau.ac.il\\

\section{Required Reading}

Guy Even and Moti Medina: \textit{Digital Logic Design}

\newpage
\part{Introduction to Discrete Math}

\section{Sets and Functions}

\begin{definition}[Universal set]
	The universal set is a set that contains all the possible objects.
\end{definition}

\begin{definition}[Set]
	A set is a collection of objects from a universal set.
\end{definition}

\begin{definition}[Subset]
	$A$ is a subset of $B$ if every element in $A$ is also an element in $B$. It is denoted as $A \subseteq B$
\end{definition}

\begin{definition}[Equal sets]
	Two sets $A$ and $B$ are said to be equal if $A \subseteq B$ and $B \subseteq A$.
\end{definition}

\begin{definition}[Strict containment]
	$A \subsetneq B \iff A \subseteq B \textnormal{ and } A \neq B$.
\end{definition}

\begin{definition}[Empty set]
	The empty set is the set that does not contain any element. It is usually denoted by $\emptyset$.
\end{definition}

\begin{definition}[Power set]
	The power set of a set $A$ is the set of all the subsets of $A$. The power set of $A$ is denoted by $P(A)$ or $2^A$.
\end{definition}

\begin{definition}[Ordered pair]
	Two objects (possibly equal) with an order (i.e., the first object and the second object) are called an ordered pair.
\end{definition}

\begin{definition}[Cartesian product]
	The Cartesian product of the sets $A$ and $B$ is the set 
	\begin{equation*}
		A \times B \triangleq \{ (a,b) | a \in A \textnormal{ and } b \in B \}
	\end{equation*}
\end{definition}

\begin{theorem}[De Morgan's Laws]\label{De Morgan's Laws}
	\begin{align*}
		\overline{A \cup B} &= \overline{A} \cap \overline{B}\\
		\overline{A \cap B} &= \overline{A} \cup \overline{B}
	\end{align*}
\end{theorem}

\begin{theorem}[Binary relation]
	A subset $R \subseteq A \times B$ is called a binary relation.
\end{theorem}

\begin{definition}[Function]
	A binary relation $R \subseteq A \times B$ is a function if for every $a \in A$ there exists a unique element $b \in B$ such that $(a,b) \in R$.
\end{definition}

\begin{definition}[Extension]
	Let $f$ and $g$ denote two functions. $g$ is an extension of $f$ if $f \subseteq g$, i.e., if every ordered pair in $f$ is also an ordered pair in $g$.
\end{definition}

\begin{definition}[Boolean function]
	A function B : {0, 1}n → {0, 1}k is called a Boolean function.
\end{definition}

\subsection{Important Boolean Functions}

\begin{definition}[$\NOT$]
	\begin{equation*}
		\NOT(x) = 1 - x
	\end{equation*}
\end{definition}

\begin{definition}[$\AND$]
	\begin{equation*}
		\AND(x) = x \cdot y
	\end{equation*}
\end{definition}

\begin{definition}[$\OR$]
	\begin{equation*}
		\OR(x) = x + y - (x \cdot y)
	\end{equation*}
\end{definition}

\begin{definition}[$\XOR$]
	\begin{equation*}
		\XOR(x) = (x + y) \bmod{2}
	\end{equation*}
\end{definition}

\section{Mathematical Induction}

\begin{theorem}
	For every $n \geq 2$, and for sets $A_1, \dots, A_n$,
	\begin{equation*}
		\overline{A_1 \cup \dots \cup A_n} = \overline{A_1} \cap \dots \cap \overline{A_n}
	\end{equation*}
\end{theorem}

\begin{proof}
	If $n = 2$, the statement is true, by \nameref{De Morgan's Laws}.\\
	Let
	\begin{equation*}
		B = A_1 \cup \dots \cup A_n
	\end{equation*}
	If possible, let
	\begin{align*}
		\overline{B} &= \overline{A_1} \cap \dots \cap \overline{A_n}
	\end{align*}
	Therefore,
	\begin{align*}
		\overline{A_1 \cup \dots \cup A_n \cup A_{n + 1}} &= \overline{B \cup A_{n + 1}}\\
		&= \overline{B} \cap \overline{A_{n + 1}}\\
		&= (\overline{A_1} \cap \dots \cap \overline{A_n}) \cap \overline{A_{n + 1}}
	\end{align*}
\end{proof}

\begin{theorem}[Pigeon-hole Principle]
	If there are $n$ holes and $m > n$ pigeons then there exists at least one hole with more than one pigeon in it. Let $f : A \to \{ 1,...,n \}$, and $|A| > n$, then $f$ is not one-to-one, i.e., there are distinct $a_1, a_2 \in A ; a_1 \neq a_2$, such that $f(a_1) = f(a_2)$.
\end{theorem}

\end{document}